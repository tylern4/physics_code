\section{Establishing an upper limit on the level of tolerable magnetic field} In the final design and implementation of the magnetic shielding, the tests show that there is going to be no reduction in signal amplitude in fields up to $30\;G$ in the transverse and longitudinal orientation. This is already beyond the maximum fields to which the PMTs will be exposed to according to the CLAS12 design requirements \cite{CLAS12FTOFstudies}. However, tests were run to note signal reduction levels as the magnetic field strength was increasted beyond $30\;G$ in each of the orientations and the point at which the signal amplitude reduced by 10\% in each of the orientations, transverse and longitudinal, was noted. Since up to such reduction levels, the time resolution is unaffected, the noted field strength serves as the conservative upper limit of the magnetic field at which the time resolution remains unaffected. Tab. 4 shows the results of such tests which establishes the conservative upper limit at $XX\;G$ and $YY\;G$ in the transverse and longitudinal orientations, respectively.

\begin{table}[H]
	\begin{center}
		\begin{tabular}{|c|c|}
			\hline
			30G-L &  \\
 			30G-T &  \\ 
 			\hline
 			35G-L &  \\
 			35G-T &  \\
 			\hline
 			40G-L &  \\
 			40G-T &  \\
 			\hline
 			45G-L & \\
 			45G-T &  \\
 			\hline
 			50G-L &  \\
 			50G-T &  \\
 			\hline
 		\end{tabular}
	\end{center}
	\caption{Signal reduction of PMT's anode signal beyond the design consideration of $30\;G$}
\end{table}
