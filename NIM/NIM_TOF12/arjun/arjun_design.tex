\subsection{Initial design considerations}
The PMT assemblies R9779-20MOD are already manufactured with a layer of mu-metal coating. It can be seen from Tab. 1 that in the transverse orientation, compared to a PMT without any mu-metal coating, the inbuilt mu-metal shielding preserves the signal amplitude to much higher levels of magnetic fields.

\begin{table}[H]
	\begin{center}
		\begin{tabular}{|c|c|l|}
			\hline
	 		& no mu-metal & with mu-metal \\
			\hline
 			10G-L & 10\% & 10\% \\
 			10G-T & 100\% & 0\% \\ 
 			\hline
 			15G-L & XX\% & XX\% \\
 			15G-T & 100\% & 0\% \\
 			\hline
 			15G-L & YY\% & YY\% \\
 			15G-T & 100\% & 0\% \\
 			\hline
 			20G-L & ZZ\% & ZZ\% \\
 			20G-T & 100\% & 10\% \\
 			\hline
 			25G-L & AA\% & AA\% \\
 			25G-T & 100\% & XX\% \\
 			\hline
 			30G-L & BB\% & BB\% \\
 			30G-T & 100\% & BB\% \\
 			\hline
		\end{tabular}
	\end{center}
	\caption{Reduction of PMT anode signal in various test configurations}
\end{table}


However, even with the inbuilt mu-metal shielding, there is a rapid deterioration of the signal past 10G-L and XXG-T and this led to considering further shielding methods.
