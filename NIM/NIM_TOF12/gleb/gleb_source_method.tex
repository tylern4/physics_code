\subsubsection{Source method}
\label{sect:Source_method}

One of the methods that can be used to determine counter resolution is so called "source" or "coordinate method" (see CLAS-NOTE 2004-016 \cite{knu-clasnote}) that is based on the fact that the light flash coordinate and arrival times of signals in two PMTs are correlated. In this method radioactive small Sr-90 $\beta$-source is placed in various positions along the bar in order to measure position-specific resolution. 

For given project this method is highly inappropriate for two main reasons: (a) large number of counters (372) with length up to four meters leads to huge amount of manual measurements; (b) the Sr-90 events
are not representative neither in penetration depth nor photon statistics, since energy of $\beta$-particles from the source is in order of MeV, while particles that will be measured in CLAS12 have energy order of GeV.

Thus this method was used for initial electronic and selected individual counters test only.
