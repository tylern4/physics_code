\section{FTOF12 requirements and high-level design}
\label{FTOF12}


{\it Important figures:\\
particle separation curves on t vs. p assuming 4separation (p$K$, $K\pi$, p$\pi$).  print quality version of some combination of these: \\
resolution curves on time resolution vs. counter length (old, new, combined) \\
resolution curves on time resolution vs. counter length per scintillator material (per advertised attenuation lengths)\\
CLAS12 geometry $\rightarrow$ FTOF12 geometry (zoom in, light guides and PMT form-factors)
stray B-field map (or just state upper bound)\\
stray B-field map (or just state upper bound)
}

The time-of-flight subsystem of the CLAS detector in Hall B was designed to allow separation of pions and kaons in the kinematic range accessible with a $6$-$GeV$ electron beam by providing time resolutions from $90\:ps$ to $160\:ps$ at the forward angles, where the most energetic particles are detected.~\cite{clastof} To reliably separate $p$, $\pi$, and $K$ in the kinematic range accessible with the proposed $11$-$GeV$ beam of the CEBAF upgrade, the FTOF detector must achieve a resolution of $80\:ps$~\cite{tdr} as illustrated in Fig.$\:$\ref{fig:partIdReq}. This assumes a $4\sigma$ time difference between two particles, thus allowing for identification of a signal in the presence of other particles with a ten-fold higher rate.
\begin{figure}[h!]
\centerline{\includegraphics[width=6cm,height=6cm]{evan/fig_evan_ftof_requirements/DeltaT.pdf}}
\caption{The three curves indicate the time differences,$\Delta t$, between $p/\pi$,$p/K$, and $\pi/K$ over the $650\;cm$ path length from the target to Panel 1B. }
\label{fig:DeltaT}
\end{figure}

\begin{figure}[h!]
\centerline{\includegraphics[width=6cm,height=6cm]{evan/fig_evan_ftof_requirements/c1_design.pdf}}
\caption{Solid line, the measured and parameterized average standard deviation $\overline{\sigma_{ToF}}$ time resolution of the existing panel-1a counters ($15\;cm$ wide, $5\;cm$ thick) fit with an assumed intrinsic electronic resolution of 40 ps ~\cite{smith1999time}. Scaling the parametrization by $\sqrt{\frac{2}{5}}$ leads to the  Long dashed line representing the new panel-1b counters (6cm wide, 6cm thick) and short dashed line the combined panel-1a and panel-1b counters time resolution. The black straight short dashed line is the design requiement for combined time resolution}
\label{fig:Design}
\end{figure}

\begin{figure}[h!]
\centerline{\includegraphics[width=6cm,height=6cm]{evan/fig_evan_ftof_requirements/c1_real.pdf}}
\caption{Solid line, the measured and parameterized average standard deviation $\overline{\sigma_{ToF}}$ time resolution of the existing panel-1a counters ($15\;cm$ wide, $5\;cm$ thick) fit with an assumed intrinsic electronic resolution of 40 ps ~\cite{smith1999time}. USC measured and parameterized average standard deviation time resolution of panel-1b leads to the  Long dashed line representing the new panel-1b counters (6cm wide, 6cm thick) and short dashed line the combined panel-1a and panel-1b counters time resolution. The black straight short dashed line is the design requiement for combined time resolution}
\label{fig:real}
\end{figure}



\begin{figure}[h!]
\centering
\mbox{\subfigure[$\:$Particle separation.]{\includegraphics[width=0.4\textwidth]{evan/fig_evan_ftof_requirements/p_max.pdf}}\quad
\subfigure[$\:$Resolution curves.]{\includegraphics[width=0.4\textwidth]{evan/fig_evan_ftof_requirements/res.pdf}}}
\caption{(a) Maximum momentum up to which particles can be separated as function of bar set number or polar angle theta. Calculations were made in assumption of 700-$cm$ path length from the target to Panel 1B.~\cite{tdr}. Various symbols represent different particle pairs:  red circles - $p$/$\pi$, purple triangles - $p$/$K$, and  blue squares - $K$/$\pi$. Curves represent combined ability of Panel 1A and Panel 1B to separate particles over momentum. Colors of the curves correspond to the colors of symbols. (b) Time resolution averaged over six same length individual bars as function of bar set number or bar length for measurements performed at USC. \label{fig:partIdReq}\label{fig:resLength}}
\end{figure}

As in the current $6$-$GeV$ FTOF detector (Panel 1A), each counter of the additional $12$-$GeV$ FTOF (Panel 1B) is composed of a long rectangular plastic scintillator with two cylindrical PMTs, one on each end, directly attached without light guides. The scintillator lengths are tightly constrained by the established six-panel FTOF geometry and the requirement that the new panels do not  restrict the CLAS12 acceptance as defined by the other detector components, but the thickness and width, $6\:cm \times 6\:cm$, are selected to optimize photon statistics, geometric matching with the photocathodes, and closest possible stacking. Compared to the Panel-1A $5\:cm \times 15\:cm$ FTOF scintillators, the $6\:cm \times 6\:cm$ scintillators increase the number of photons produced by a factor of 6/5; the increased ratio of photocathode area to scintillator exit window area increases the number of photons that reach the photocathode by at least a factor of 25/12.  Thus, disregarding that the area ratio factor acts on the number of photons after light attenuation, the Panel-1B scintillator geometry increases the number of photons reaching the photocathode by a factor of about 5/2 and, therefore, improves the resolution by a factor of $\sqrt{2/5}$, neglecting the resolution of any contributions that are independent of light level.

With a resolution of better than $150\:ps$ for the longest counters of Panel 1A, the Panel-1B counters must achieve resolutions better than $95\:ps$ for the combined resolution goal of $80\:ps$ to be reached (Fig.$\:$\ref{fig:resLength}). Preliminary prototype results exceed this requirement.
